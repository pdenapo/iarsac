% documentación de IARSAC en LaTeX
% use:
%  pdflatex iarsac-doc.tex 
% para generar la documentación en PDF

\documentclass[spanish]{article}
\usepackage[utf8]{inputenc}
\usepackage{amssymb}
\usepackage{html}
\IfFileExists{url.sty}{\usepackage{url}}
                      {\newcommand{\url}{\texttt}}

\makeatletter
%%%%%%%%%%%%%%%%%%%%%%%%%%%%%% Textclass specific LaTeX commands.
 \newcommand {\theoremstyle} [1] { }
 \newtheorem{thm}{Teorema}[section]

\usepackage{babel}
\addto\extrasspanish{\bbl@deactivate{~}}
\makeatother
\begin{document}

\title{IARSAC: Implementación del Algoritmo RSA en C++ - Versión 0.2}


\author{Pablo De Nápoli}

\maketitle

\section{Introducción}

\subsection*{Propósito de IARSAC}

RSA es uno de los algoritmos de criptografía de clave pública (o de
claves asimétricas) más utilizados. Fue desarrollado por Ronald Rivest,
Adi Shamir, and Leonard en 1977 (\cite{Rsa78}. El nombre de este
algoritmo corresponde a las letras iniciales de cada uno de sus 
creadores. Para una descripción matemática, ver \cite{K}.

IARSAC es una implementación del algoritmo RSA en C++ (como una clase).
Está diseñado para ser una implementación minimalista con propósitos
didácticos (para aquellos que deseen aprender como funciona el algoritmo
RSA).

IARSAC no es una implementación segura, y no se recomienda su uso
para ninguna aplicación donde la seguridad de sus datos sea importante
(para tales aplicaciones usted puede utilizar 
\htmladdnormallink{GPG: The GNU Privacy Guard}{http://www.gnupg.org/})

\subsection*{Requerimientos}

IARSAC utiliza la librería \htmladdnormallink{NTL}{http://www.shoup.net/ntl/} 
(una librería para teoría de números), que también se distribuye bajo 
la Licencia Pública GNU.

Adicionalmente (dependiendo de como compile NTL) es posible que necesite
la librería GNU MP (Multiple Precision Library).

IARSAC solamente ha sido probado bajo GNU/Linux, y es probable que la 
implementación de la función random\_seed (que usa el dispositivo /dev/random, 
el generador de números aleatorios del kernel) para inicializar 
el generador de números aleatorios de NTL sea no portable a otros
sistemas (pero podría implementarse de otra forma).

Para compilarlo (yo lo he probado usando gcc-3.3.2, ntl 5.3 y gmp 4.1.2) 
hay que usar una orden como

\begin{verbatim}
g++ iarsac.cpp  -fpermissive -lntl -lgmp -lm -o iarsac 
\end{verbatim}

\subsection*{Licencia}

IARSAC se distribuye bayo la Licencia Pública GNU (GNU General Public
Licence), de la Free Software Foundation (versión 2 o posterior).
No existe ningún tipo de garantía (ni siquiera garantías implícitas
de utilidad o adecuación para algún propósito).

\subsection*{Patentes }

RSA fue durante un tiempo un algoritmo patentado (U.S. Patent 4,405,829),
pero esta patente expiró el 2 de septiembre de 2000 (ver 
\htmladdnormallink{http://www.rsasecurity.com/rsalabs/faq/6-3-1.html}
{http://www.rsasecurity.com/rsalabs/faq/6-3-1.html}).

\section{Uso de IARSAC}

IARSAC acepta las siguientes opciones

\begin{itemize}

\item

\begin{verbatim}
-g --generar-claves-para:   
\end{verbatim}

genera las claves pública y privada para un usuario, 
que se graban en los archivos nombre\_de\_usuario.clave\_publica y 
nombre\_de\_usuario.clave\_privada. Se puede especificar una longitud de clave 
con la opción -l (la longitud de clave por defecto es de 100)

ejemplo:

\begin{verbatim}
iarsac -l 80 -g Alicia -v 
\end{verbatim}

produce una salida del tipo

\begin{verbatim}
Generando claves RSA para Alicia
longitud de clave=80 bits 
Primos elegidos 
p=961018495041910433981507
q=655540418525339563040093
clave_publica=772934981430496353782275
n=629986466450365929669563511644626552491061560151
phi_n=629986466450365929669561895085712985241064538552
clave_privada=599711234594355997158663147623933028514632134163
\end{verbatim}

En este ejemplo las claves se guardan en los archivos Alicia.clave\_publica y
Alicia.clave\_privada en el directorio actual.

\item 
\begin{verbatim}
-c --cifrar-para
\end{verbatim}

Cifra la entrada (del standar input) y envia el texto encriptado al 
standard output, utilizando la clave pública del usuario especificado
(que debe estar contenida en el archivo nombre\_de\_usuario.clave\_publica en
el directorio actual). Como el texto encriptado 
suele ser no legible para los humanos, es conveniente redireccionar la salida 
a un archivo.

ejemplo:
\begin{verbatim}
iarsac -c Alicia < mi_texto_plano.txt > mi_texto_encriptado.bin
\end{verbatim}

En este sentido, IARSAC sigue los estándares del mundo de Unix.

Si se desea enviar los mensajes encriptados por e-mail, se
puede combinarlo con el comando uuencode por medio de una tubería
(pipe):

ejemplo:
\begin{verbatim}
iarsac -c Alicia < mi_texto_plano.txt | uuencode - > mi_texto_encriptado.bin
\end{verbatim}

uuencode y uudecode forman parte el paquete GNU sharutils.

\item
\begin{verbatim}
-d --descifrar-para: descifra un mensaje 
\end{verbatim}

Descifra la entrada (del estandard input), y envia la salida al
estandard output, utilizando la clave privada del usuario especificado, que
debe estar contenida en el archivo usuario.clave\_privada en el directorio
actual. Normalmente es conveniente utilizar redireccion, 
por ejemplo Alicia para leer el mensaje encriptado anteriormente 
usaria:

Si se desea desencriptar un mensaje codificado con uuencode como en el
ejemplo anterior usar uudecode y una tubería (pipe):

\begin{verbatim}
uudecode mi_texto_encriptado.txt | iarsac -d Alicia > mi_texto_desencriptado.txt
\end{verbatim}


\item 
\begin{verbatim}
-h --help
\end{verbatim}

Muestra un mensaje de ayuda sobre las opciones de IARSAC.

\item
\begin{verbatim} 
-L --license muestra información sobre la licencia GPL 
\end{verbatim}

Muestra información sobre la licencia GPL 

\item 
\begin{verbatim}
-l --longitud-de-clave 
\end{verbatim}

Permite especificar la longitud de clave a utilizar.

\item
\begin{verbatim}
-v --verbose 
\end{verbatim}

Muestra información adicional sobre el funcionamiento interno del programa. Es
util para propósitos de debugging (corrección de errores) o para analizar el
funcionamiento del programa con fines didácticos.

Para mostrar dicha información se usa el standard error (con lo que no 
afecta la redirección del standard output)

\end{itemize}

\section{API de IARSAC}

\begin{verbatim}

El algoritmo RSA es implementado en IARSAC como la clase encriptador_rsa. 
Esto permitiría utilizar dicho algoritmo en otros programas.

\begin{verbatim}
class encriptador_rsa
{

public:
  
  void generar_claves(long longitud);
  
  void cifrar(istream& texto_plano,ostream& texto_encriptado);

  void descifrar(istream& texto_encriptado,ostream& texto_plano);
  
  void establecer_parametros(); ; 
 
  ZZ n;
  ZZ clave_publica;
  ZZ clave_privada;
 
private: 

  // n sera el tamaño del "alfabeto" en el cual vamos a cifrar o
  // descifrar

  long longitud_de_clave;
  long long_bloque_plano;
  long long_bloque_encriptado;
 
};

\end{verbatim}

NOTA: esta API está en desarrollo y puede cambiar en futuras versiones de
iarsac.

Describiremos a continuación los distintos métodos de esta clase:

\begin{itemize}

\item encriptador\_rsa::generar\_claves:

Este método genera las claves pública y privada, y el entero n. 
Toma como parámetro la longitud de la clave que se va a generar.

Para generar dichas claves hacemos lo siguiente:

\begin{enumerate} 

\item Elegimos dos números primos al azar $p$ y $q$, de la longitud especificada 
(con la función GenPrime\_ZZ de la libreria NTL)

\begin{verbatim}
 p = GenPrime_ZZ(longitud);
 q = GenPrime_ZZ(longitud);
\end{verbatim}

\item Calculamos $n=pq$ y $\varphi(n)=(p-1)(q-1)$ ($\varphi(n)$ es la función
de Euler que cuenta cuantos números entre 1 y n son coprimos con $n$)

\begin{verbatim}
 n = p*q;
 phi_n = (p-1)*(q-1);
\end{verbatim} 

Vamos con aritmética módulo $n$ (es decir en $\mathbb{Z}_n$)

\item Generamos un número al azar entre 1 y n (es decir un elemento de
$\mathbb{Z}_n$) que no tenga factores en común (sea coprimo) con $n$. 
Dicho número será la clave publica.

\begin{verbatim}
 do {
	 clave_publica = RandomLen_ZZ(longitud)%n;
 } while (GCD(clave_publica,phi_n)!=1);
\end{verbatim}

\item La clave privada es el inverso módulo n de la clave pública:

\begin{verbatim}
  clave_privada = InvMod(clave_publica,phi_n);  
\end{verbatim}

\end{enumerate}

\item encriptador\_rsa::establecer\_parametros:

Este método establece los valores de los parámetros longitud\_bloque\_plano 
y longitud\_bloque\_encriptado. 

\item encriptador\_rsa::cifrar

Este método encripta un texto proveniente de un input stream 
(texto\_plano), y devuelve como salida el texto desencriptado
en otro stream (texto\_encriptado).

El texto plano se parte en bloques del taman~o bloque\_plano (determinado 
en función de $n$ por el método establecer\_parametros)

Cada bloque se transforma a numero por la función

\begin{verbatim}
BytesFromZZ((unsigned char*)bloque_encriptado,numero_encriptado,long_bloque_encriptado);
\end{verbatim}

después se encripta mediante la instrucción 

\begin{verbatim}
numero_encriptado = PowerMod(numero_plano,clave_publica,n);
\end{verbatim}

finalmente el número encriptado se convierte a bytes

\begin{verbatim}
BytesFromZZ((unsigned char*)bloque_encriptado,numero_encriptado,long_bloque_encriptado);
\end{verbatim}

En el stream de salida se escriben el numero de bytes del bloque plano leido 
como cabecera, y el bloque encriptado.

\item encriptador\_rsa::descifrar

Este método desencripta un texto proveniente de un input stream 
(texto\_encriptado), y devuelve como salida el texto desencriptado
en otro stream (texto\_plano).

Se lee cada bloque proveniente del stream (teniendo en cuenta la cabecera
de bloque que indica cuantos bytes forman el bloque)

Cada bloque se convierte en un número:

\begin{verbatim}
numero_encriptado = ZZFromBytes((unsigned char*)bloque_encriptado,long_bloque_encriptado);
\end{verbatim}

se lo desencripta mediante la instrucción:

\begin{verbatim}
numero_plano = PowerMod(numero_encriptado,clave_privada,n);
\end{verbatim}

y después se lo transforma en bytes mediante la instrucción

\begin{verbatim}
BytesFromZZ((unsigned char*)bloque_plano,numero_plano,long_bloque_plano);
\end{verbatim}

\end{itemize}

\section{Licencia de Este Documento}

Copyright (C) 2004-2012 Pablo Luis De Nápoli

Se garantiza el permiso para copiar, distribuir y/o modificar este documento
bajo los términos de la Licencia de Documentación Libre GNU, versión 1.1 
(\htmladdnormallink{GNU Free Documentation License}{http://www.gnu.org/copyleft/fdl.html}, Version 1.2 ) 
o cualquier otra versión 
posterior publicada por la Free Software Foundation; este documento se 
presenta sin Secciones Invariables (no Invariant Sections), sin Textos de 
Tapa (no Front-Cover Texts) y sin Textos de Contratapa (no Back-Cover Texts). 


\begin{thebibliography}{Rsa-Faq}
\bibitem[Rsa78]{Rsa78}R.L. Rivest, A. Shamir, and L.M. Adleman, A method for 
obtaining digital signatures and public-key cryptosystems, 
Communications of the ACM
(2) 21 (1978), 120-126.
\bibitem[Rsa-Faq]{Rsa-Faq} 
\htmladdnormallink 
{RSA Laboratories' Frequently Asked Questions About Today's Cryptography.}
{http://www.rsasecurity.com/rsalabs/faq/6-3-1.html}
\bibitem [Ko87]{K} N. Koblitz. A course in Number Theory and Crytography. 
Springer Verlag, (1987).

\end{thebibliography}


\end{document}
